\section{Business Goals}
The goal of the business is to provide customers with reliable, fast and relevant public transportation services,
all whilst making the running costs (infrastructures, staff, etc.) as low as possible to make as much profit as possible.

In order to make the service fast and relevant, the company tracks what stops seem to be the most used by customers,
either through driver feedback or through the use of the website, and sees what sections and portions of the main lines 
are used the most at certain times of the day, adapting the routes and services available at certain times of the day according 
to the user needs (examples of adaptation would be school holiday timetables and term timetables, as well as not running the 
whole line at certain times of the day).

To make sure that the service is reliable, it constantly verifies if the rotas are properly allocated (no drivers assigned to two different 
routes at the same time) and checks against traffic patterns and provisions the time that it takes to go between two stops, reflecting
that difference in the timetable.

To avoid costs, the company makes sure that all the drivers have their hours allocated properly (not making less or more hours than 
what they are contracted for), as well as some other systems that are used to make the service fast and reliable (an example would be 
the adaptation of the lines and routes to the time of the day, by not having too many buses running at a time where they aren't needed).

To provide a better service to their customers, the company now wants to provide the customers with an online service where they can see 
the updated timetables for their favourite services at all times, provide staff with a service that makes it easier for them to update the timetables 
and make rotas, where all the information is centralized and where the computer systems eliminate the tedious work of consistency check.