\chapter{Background \& Rationale}

\section{Enterprise Model}

A clerk in the head office (or operational manager) sets the routes and times they want for the bus lines.

Drivers are assigned to the routes, reducing the total time of the route to their contracted hours. The closer it is to zero (hours left to allocate) the better.

Customers can look at the timetable for the line, without knowing who the driver is, and choose the bus they want to take.

The customers board the bus, pay the driver, or show them a previously bought and still valid ticket, and take a seat.

\section{Business Goals}

The goal of the business is to provide customers with a reliable, fast and relevant public transportation service. All whilst making the running costs (infrastructure, staff, etc.) as optimal as possible.

In order to make the service fast and relevant, the company tracks what stops seem to be the most used by customers, either through driver feedback or through the use of the website. They can then check what sections of the main lines are used the most at various times of the day, adapting the routes and services available. For example, increased peak time services during school holidays.

To avoid unnessary costs, the company makes sure that all drivers have their hours allocated properly - ensuring they are not 