\documentclass{report}
    \usepackage{enumitem}
    \usepackage{listings}
    \usepackage{xcolor}
    
    \lstdefinestyle{sharpc}{language=[Sharp]C, frame=lr, rulecolor=\color{blue!80!black}}

\begin{document}
    \chapter{Start a project}
    \begin{itemize}
        \item Create a folder for the project
        \item Open CMD on the folder (or terminal)
        \item Make sure that .NET Core SDK is installed with the dotnet --version command
        \begin{itemize}
            \item If it is installed, move to the next step
            \item If it isn't installed google .net core sdk and follow the installation procedures
        \end{itemize}
        \item To create a new MVC project on that file, run: dotnet new mvc
    \end{itemize}

    \chapter{Write the code for the database (Code-First Approach)}
    \section{Add the Entity Framework Core packages to the project}
    \begin{itemize}
        \item Open Visual Studio Code on the folder of the project you just started
        \item Open the terminal inside VS Code
        \item Run dotnet add package Microsoft.VisualStudio.Web.CodeGeneration.Design
        \item Run dotnet add package Microsoft.EntityFrameworkCore.Tools.DotNet
    \end{itemize}

    \section{Add the classes representing the tables}
    In the Code First approach, each of the classes that will be created in the Models
    folder will represent an entity in the database.
    \subsection{Add the class file}
    \begin{itemize}
        \item In the file explorer, right-click on the "Models" folder and select add file
        \item Add the name of the Entity to the file and end with the .cs extension
    \end{itemize}

    \subsection{Add code to the class}
    After creating the class, it needs to be added to the Models namespace and have its attributes
    describing the attributes of the entity. The general structure for the class is the following:

    \lstset{style=sharpc}
    \begin{lstlisting}
        using System;
        using System.Collections.Generic;

        namespace [ProjectName].Models
        {
            public class [ClassName]
            {
                //Add your entity attributes here. Integers would be ints
                //Strings would be string
                //Date Time types would be DateTime
                public int ID {get; set;}

                //This class doesn't have any constructor

                //In order to establish a one-to-many relationship where the
                //current entity has many entries in another table, do:
                public ICollection<[OtherEntityName]> [CollectionName] {get; set;}
            }
        }
    \end{lstlisting}

    \section{Create the Database Context}
    The database context is the main class that coordinates the Entity Framework
    functionality for a given data model. We create this class by deriving it from
    the Microsoft.EntityFrameworkCore.DbContext class. In the code for this class
    we must specify which entities are included in the data model. We can also
    customize certain Entity Framework behaviour.

    \begin{itemize}
        \item In the project folder, create a folder named Data
        \item In the Data folder create a new file named [ProjectName]Context.cs
        and insert the following code:
    \end{itemize}

    \lstset{style=sharpc}
    \begin{lstlisting}
        using [ProjectName].Models;
        using Microsoft.EntityFrameworkCode;

        namespace [ProjectName].Data
        {
            public class [ProjectName]Context : DbContext
            {
                public [ProjectName]Context(DbContextOptions<[ProjectName]Context> options) : base(options)
                {
                }

                public DbSet<[EntityName]> [EntityName] {get; set;}
                public DbSet<[EntityName]> [EntityName] {get; set;}
                //... Continue with the DbSets for as many entities as
                //necessary for the project
            }
        }
    \end{lstlisting}

    This code creates a DbSet property for each entity set. In Entity Framework terminology,
    an entity set typically corresponds to a database table, and an entity corresponds to a row in the table.

    When the database is created, EF creates tables that have names the same as the DbSet property names.
    Property names for collections are typically plural, however, developers disagree about whether table
    names should be pluralized or not. If you want the name of your tables to be in the singular, you have
    to add the following code in DbContext to override the default behaviour of EF:
    \lstset{style=sharpc}
    \begin{lstlisting}
        protected override void OnModelCreating(ModelBuilder modelBuilder)
        {
            modelBuilder.Entity<EntityName>().ToTable("EntityName");
            //Repeat for the other entities
        }
    \end{lstlisting}

    \section{Register the context with dependency injection}
    ASP.NET Core implements dependency injection by default. Services such as the
    EF database context are registered with dependency injection during application
    startup. Components that require these services (such as MVC controllers) are
    provided these services via constructor parameters. To register the database
    context as a service, we need to:
    \begin{itemize}
        \item Open Startup.cs
        \item Add the following code in the ConfigureServices method:
    \end{itemize}

    \lstset{style=sharpc}
    \begin{lstlisting}
        services.AddDbContext<[DbContextName]>(options =>
        options.UseSqlite("Data Source=[ProjectName].db"));
    \end{lstlisting}

    In order for the DbContext to be recognized, as well as the UseSqlServer options, we need to
    add the following using statements at the beginning of Startup.cs:
    \lstset{style=sharpc}
    \begin{lstlisting}
        using [ProjectName].Data;
        using Microsoft.EntityFrameworkCore;
    \end{lstlisting}

    \chapter{Add code to initialize the database with test data}
    After the previous steps, Entity Framework will create an empty database for us.
    In this chapter we will write a method that is called after the database is created
    in order to populate it with test data.

    Here we will use the EnsureCreated method to automatically create the database.
    In order to create and populate the database we have to:
    \begin{itemize}
        \item Create a file DbInitializer.cs in the Data folder
        \item Insert the following code:
    \end{itemize}

    \lstset{style=sharpc}
    \begin{lstlisting}
        using [ProjectName].Models;
        using System;
        using System.Linq;

        namespace [ProjectName].Data
        {
            public static class DbInitializer
            {
                public static void Initialize([DbContextName] context)
                {
                    context.Database.EnsureCreated();

                    //Before populating, make sure there is no data already
                    if(context.[EntityName].Any())
                    {
                        return; //DB has been seeded
                    }

                    var [entityName] = new [EntityName][]
                    {
                        new [EntityName]{//add code here for the different attributes}
                        //Add more entries to the table here
                    };
                    foreach([EntityName] n in [varPreviouslyCreated])
                    {
                        context.[EntityName]s.Add(n);
                    }
                    //Do the previous 2 structures (var and foreach)
                    //for the remaining entities of the table

                    context.SaveChanges();
                }
            }
        }
    \end{lstlisting}

    The previous code checks if there are any members of one of the
    entities in the database, and if not, it assumes that the database
    is new and needs to be seeded with test data. It loads test data
    into arrays rather than List<t> collections to optimize performance.

    After doing this, we need to head to Program.cs and modify the main
    method to do the following on application startup:
    \begin{itemize}
        \item Get a database context instance from the dependency injection
        container
        \item Call the seed method, passing to it the context
        \item Dispose the context when the seed method is done
    \end{itemize}

    The end result should look something like this:
    \lstset{style=sharpc}
    \begin{lstlisting}
        using Microsoft.Extensions.DependencyInjection;
        using [ProjectName].Data;
        
        //Lines not shown for brevity

        public static void Main(string[] args)
        {
            var host = BuildWebHost(args);

            using(var scope = host.Services.CreateScope())
            {
                var services = scope.ServiceProvider;
                try
                {
                    var context = services.GetRequiredService<[DbContextName]>();
                    DbInitializer.Initialize(context);
                }
                catch(Exception ex)
                {
                    var logger = services.GetRequiredService<ILogger<Program>>();
                    logger.LogError(ex, "An error occurred while seeding the database.");
                }
            }

            host.Run();
        }
    \end{lstlisting}

    Now, the first time we will run the application, the database will be created and seeded with
    test data. Whenever we change our data model, we can delete the database, update the seed method,
    and start afresh with a new database the same way.

    \chapter{Create a controller and views}
    Unfortunately, as of now, ASP.NET Core does not allow for automatic
    scaffolding of views and controller for our database model using Visual Studio
    Code. Therefore we will need to hand code them, which will allow us to have
    a better understanding of how controllers and views are built and give us better
    customization options, but will, unfortunately, cost us more time.

    \section{Adding a Controller}
    Controllers are classes that handle browser requests. They retrieve model data and call view templates
    that return a response. In an MVC app, the controller handles and responds to user input and interaction.
    The controller handles route data and query-string values, and passes them to the model.
    To add a controller in Visual Studio Code, we have to:
    \begin{itemize}
        \item Right-click in the Controllers folder and select New File
        \item Write the name of the new controller (usually corresponds to the
        name of an entity) and end with the .cs extension
    \end{itemize}
    In a controller class, every public method in a controller is callable as an HTTP endpoint.
    An HTTP endpoint is a targetable URL in the web application and combines the protocol
    used: HTTP, the network location of the webserver and the target URI.

    The first comment stated before a public method usually states the pathway to the method.

    MVC invokes controller classes (and the action method within them) depending on the incoming URL.
    The default URL routing logic used by MVC uses a format like this to determine what code to invoke:
    /[Controller]/[ActionName]/[Parameters].


    We can modify the code to pass parameter information from the URL to the controller.
    By adding parameters to the action method, we specify the names of the variables
    and types that are being expected to be in the parameters section of the URL. In
    order to pass those parameters, we need to add the right name and type of value
    in the URL.

    \section{Adding a view}
    We use Razor view template files to cleanly encapsulate the process
    of generating HTML responses to a client We create a view template file using Razor.
    Razor-based template files have a .cshtml file extension. They provide an elegant way
    to create HTML output using C#.

    If we want a controller to return a view to the browser, we have to add
    the following code:
    
    \lstset{style=sharpc}
    \begin{lstlisting}
        public IActionResult Index()
        {
            return View();
        }
    \end{lstlisting}

    The preceding code returns a View object. It uses a view template to
    generate an HTML response to the browser. Controller methods (also known as
    action methods) such as the Index method above, generally return an IActionResult
    (or a class derived from ActionResult), not a type like string.

    After asking the controller to return the view, we now need to create
    the view template that will generate the HTML for our view. We add a
    view to a controller by doing:
    \begin{itemize}
        \item Add a new folder in Views with the name of the controller class
        (without the Controller part)
        \item Add a new file to the previously created folder with the name
        of the method and ending in .cshtml
    \end{itemize}

    We can then write the html mixed with UI code to generate the view.

    \section{Changing views and layout pages}
    If we tap the menu links, we can easily notice that each page shows the
    same menu layout. The menu layout is implemented in the
    Views/Shared/_Layout.cshtml file.

    Layout templates allow us to specify the HTML container layout of our side in one
    place and then apply it across multiple pages in our site. The RenderBody call in
    the Layout.cshtml file is a placeholder where all the view-specific pages we create
    show up, wrapped in the layout page. For example, if we select the About link,
    it is the Views/Home/About.cshtml view that is rendered inside the RenderBody method.

    \section{Change the title and menu link in the layout file}
    We can change the contents of the title element. We can change the anchor text in the
    layout template to the name of our app and the controller from Home to whatever name we wish.

    If we want to change the title of the app, we have to go to the line of code that looks like this:

    \lstset{style=sharpc}
    \begin{lstlisting}
        <title>@ViewData["Title"] - [AppName]</title>
    \end{lstlisting}

    Then, inside the navbar-header div, we change the main title that appears in the
    web page on the top left corner:
    \lstset{style=sharpc}
    \begin{lstlisting}
        <a asp-area="" asp-controller="[MainControllerName]" asp-action="Index" class="navbar-brand">[App Name]</a>
    \end{lstlisting}

    By using the _Layout.cshtml file for our main layout, it allows us to make changes
    once in the whole web app, saving us time and hassle.

    The Views/_ViewStart.cshtml file brings in the Views/Shared/_Layout.cshtml
    file to each view. We can use the Layout property to set a different layout
    view, or set it to null so no layout file will be used.

    We can change the title of the Index view as well.
    When we open the Views/Home/Index.cshtml file, there are two places
    to make a title change:
    \begin{itemize}
        \item The text that appears in the title of the browser
        \item The secondary header (<h2> element)
    \end{itemize}

    To see what does what, we can make them slightly different
    to see which bit of code changes which part of the app.

    \section{Passing Data from the Controller to the View}
    Controller actions are invoked in response to an incoming URL
    request. A controller class is where we write code that handles
    the incoming browser requests. The controller retrieves data from
    a data source and decides what type of response to send back to
    the browser. View templates can be used from a controller to generate
    and format an HTML response to the browser.

    Controllers are responsible for providing data required in order for a
    view template to render a response. A best practice: View templates
    should not perform business logic or interact with a database directly.
    Rather, a view template should work only with the data that's provided
    to it by the controller. Maintaining this "separation of concerns" helps
    keep our code clean, testable and maintainable.

    If we have a controller receiving values in the URL as parameters, instead
    of rendering them as a string back to the browser, we can change the controller
    to use a view template instead. The view template will generate a dynamic
    response, which means that we need to pass the appropriate bits of data from
    the controller to the view in order to generate the response.
    We can do this by having the controller put the dynamic data (parameters) that
    the view template needs in a ViewData dictionary that the view template can then
    access.

    Let's imagine a HelloWorldController file with a welcome method
    that receives parameters that need to be sent to the view.
    That method receives a name and ID parameters and then outputs the
    values directly to the browser. We can now change the Welcome method
    to add a Message and NumTimes value to the ViewData dictionary. The
    ViewData dictionary is a dynamic object, which means we can put whatever
    we want in to it; the ViewData object has no defined properties until we
    put something inside it. The MVC model binding system automatically maps
    the named parameters (name and numTimes) from the query string in the
    address bar to parameters in our method.

    The finished HelloWorldController would look like this:
    \lstset{style=sharpc}
    \begin{lstlisting}
        using Microsoft.AspNetCore.Mvc;
        using System.Text.Encodings.Web;

        namespace MvcMovie.Controllers
        {
            public class HelloWorldController : Controller
            {
                public IActionResult Index()
                {
                    return View();
                }

                public IActionResult Welcome(string name, int numTimes = 1)
                {
                    ViewData["Message"] = "Hello " + name;
                    ViewData["NumTimes"] = numTimes;

                    return View();
                }
            }
        }
    \end{lstlisting}

    The ViewData dictionary object contains data that will be passed to the view.

    We can then create a Welcome view template named Views/HelloWorld/Welcome.cshtml
    We'll create a loop in the Welcome.cshtml view template that displays the Message
    for NumTimes. We replace the contents of Views/HelloWorld/Welcome.cshtml with the
    following:

    \lstset{style=sharpc}
    \begin{lstlisting}
        @{
            ViewData["Title"] = "Welcome";
        }

        <h2>Welcome</h2>

        <ul>
            @for (int i = 0; i < (int)ViewData["NumTimes"];i++)
            {
                <li>@ViewData["Message"]</li>
            }
        </ul>
    \end{lstlisting}

    The data is taken from the URL and passed to the controller using the MVC
    model binder. The controller packages the data into a ViewData dictionary
    and passes that object to the view. The view renders the data as HTML to the browser.

    In the sample above, we used the ViewData dictionary to pass data from the controller
    to a view. Later we will also use a view model to pass data from a data from a controller
    to a view. The view model approach of passing data is generally much preferred
    over the ViewData dictionary approach.

    \chapter{Scaffolding Controllers for your data model}
    In order to generate controllers for your data model, you have to open the terminal
    and run the following command for each of your DbContexts:
    \begin{itemize}
        \item dotnet restore
        \item dotnet aspnet-codegenerator controller -name [ControllerName] -m [ModelName] -dc [DbContext] --relativeFolderPath Controllers --useDefaultLayout --referenceScriptLibraries
    \end{itemize}

    With these commands, the scaffolding engine creates the following:
    \begin{itemize}
        \item A controller for the model
        \item CRUD (Create, Delete, Details, Edit and Index) Razor view pages
        for the model
    \end{itemize}
    
    The automatic creation of CRUD action methods and views is known as scaffolding.
    If you ever run into problems during the scaffolding process, try deleting the obj folder
    from the project, and running both the commands again.

    The controller takes a DbContext object as a constructor parameter.
    ASP.NET dependency injection will take care of passing an instance
    of the DbContext into the controller. We configured that in the
    Startup.cs file earlier.

    The controller contains an Index action method, which displays all
    the data for that entity in the database. The method gets a list of all
    the data in the entity set by reading the appropriate property of the
    database context, by the method:
    \begin{itemize}
        \item return View(await _context.[EntityName].ToListAsync())
    \end{itemize}

    The view associated with that model displays the list in a table,
    using @foreach and @Html.DisplayFor Razor statements.

    \chapter{Working with SQLite}
    The DbContext object handles the task of connecting to the database
    and mapping the objects to database records. The database context is
    registered with the Dependency Injection container in the ConfigureServices
    method in the Startup.cs.

    You can use DB Browser or other apps to manage SQLite databases.

    \chapter{Controller methods and views}
    In order to change the name and format of data when it gets
    displayed in the views, we have to go to the model for which we
    want to optimize the display of data, and do the following changes:
    \begin{itemize}
        \item Add using System.ComponentModel.DataAnnotations; to the used dependencies
        \item Add the data annotation that you wish to apply to the
        data.
    \end{itemize}

    In data annotations, the display attribute specifies what to display for the name
    of a field. The DataType attribute specifies the type of the data, so the time
    information store in the field is not displayed.

    The Edit, Details and Delete links are generated by the Core MVC Anchor Tag Helper
    in the Index view of an entity. Tag Helpers enable server-side code to participate
    in creating and rendering HTML elements in Razor files.
    The AnchorTagHelper dynamically generated the HTML href attribute value from the
    controller action method and route id. We use View Source from our favorite browser
    or use the developer tools to examine the generated markup.

    Here is a sample of a GET edit method in a controller that fetches the movie and populates
    the edit form generated by the Edit.cshtml Razor file:
    
    \lstset{style=sharpc}
    \begin{lstlisting}
        public async Task<IActionResult> Edit(int? id).
        {
            if(id == null)
            {
                return NotFound();
            }

            var data = await _context.[EntityName].SingleOrDefaultAsync(m => m.ID == id);
            if(data == null)
            {
                return NotFound();
            }

            return View(data);
        }
    \end{lstlisting}

    The next sample method shows a HTTP POST Edit method, which processes
    the posted values to change Movies details:
    \lstset{style=sharpc}
    \begin{lstlisting}
        //Post: Movies/Edit/5
        [HttpPost]
        [ValidateAntiForgeryToken]
        public async Task<IActionResult> Edit(int id, [Bind("ID,Title,ReleaseDate,Genre,Price")] Movie movie)
        {
           if(id != movie.ID)
           {
               return NotFound();
           }
           if(ModelState.IsValid)
           {
               try
               {
                   _context.Update(movie);
                   await _context.SaveChangesAsync();
               }
               catch(DbUpdateConcurrencyException)
               {
                   if(!MovieExists(movie.ID))
                   {
                       return NotFound();
                   }
                   else
                   {
                       throw;
                   }
               }
               return RedirectToAction("Index");
           }
           return View(movie);
        }
    \end{lstlisting}

    The [Bind] attribute is one way to protect against over-posting. We
    should only include properties in the [Bind] attribute that we want
    to change. This Edit action is also preceded by the [HttpPost] attribute.

    The HttpPost attribute specifies that this Edit method can be invoked
    only for POST requests. We could apply the [HttpGet] attribute to the
    first edit method, but that's not necessary because [HttpGet] is the
    default.

    The ValidateAntiForgeryToken attribute is used to prevent forgery of a
    request and is paired up with an anti-forgery token generated in the
    edit view file. The edit view file generates the anti-forgery token with
    the Form Tag Helper: form asp-action="Edit".

    The Form Tag Helper generates a hidden anti-forgery token that must match
    the [ValidateAntiForgeryToken] generated anti-forgery token in the Edit
    method of the Movies controller.

    The HttpGet Edit method takes the movie ID parameter, looks up the movie
    using the Entity Framework SingleOrDefaultAsync method, and returns the
    selected movie to the Edit view. If a movie cannot be found, NotFound (HTTP
    404) is returned.

    When the scaffolding system creates the Edit view, it examined the Movie class
    and created code to render <label> and <input> elements for each property of
    the class.

    When looking at a scaffolded view, we notice that a reference to the model is
    one of the first statements at the top of the file. The reference specifies that
    the view expects the model for the view template to be of a certain type.

    The scaffolded code uses several Tag Helper method to streamline the HTML
    markup. The Label tag Helper displays the name of the field. The Input Tag
    Helper renders an HTML input element. The Validation Tag Helper displays any
    validation messages associated with that property.

    Back to the controller, the [ValidateAntiForgeryToken] attribute validates the
    hidden XSRF token generated by the anti-forgery token generator in the Form Tag
    Helper.

    The model binding system takes the posted form values and creates a Movie object
    that's passed as the movie parameter. The ModelState.IsValid method verifies that
    the data submitted in the form can be used to modify (edit or update) a Movie object.
    If the data is valid it's saved. The updated (edited) movie data is saved to the
    database by calling the SaveChangesAsync method of database context. After saving
    the data, the code redirects the user to the Index action method of the
    MoviesController class, which displays the movie collection, including the changes
    that have just been made.

    Before the form is posted to the server, client side validation checks any validation
    rules on the fields. If there are any validation errors, an error message is displayed
    and the form is not posted. If JavaScript is disabled, we won't have client side
    validation but the server will detect the posted values that are not valid, and the form
    values will be redisplayed with error messages.

    All the HttpGet methods in the movie controller follow a similar pattern. They get a
    movie object (or list of objects), and pass the object (model) to the view. The Create
    method passes an empty movie object to the Create view. All the methods that create, edit,
    delete or otherwise modify data do so in the [HttpPost] overload of the method. Modifying
    data in an HTTP GET method is a security risk. Modifying data in an HTTP GET method also
    violates HTTP best practices and the architectural REST pattern, which specifies that GET
    requests should not change the state of the application. In other words, performing a GET
    operation should be a safe operation that has no side effects and doesn't modify our
    persisted data.

    \chapter{Adding Search}
    In this section, we will work with examples of a movie renting web app and add search
    capability to the Index action method that lets us search movies by title.
    In order to look for movies only by title, we can modify the Index view
    in the following manner:
    \lstset{style=sharpc}
    \begin{lstlisting}
        public async Task<IActionResult> Index(string searchString)
        {
            var movies = from m in _context.Movie select m;

            if(!String.IsNullOrEmpty(searchString))
            {
                movies = movies.Where(s => s.Title.Contains(searchString));
            }

            return View(await movies.ToListAsync());
        }
    \end{lstlisting}

    The first line of the Index action method creates a LINQ query to select
    the movies. The query is only defined at this point, it has not been run
    against the database. If the searchString parameter contains a string, the
    movies query is modified to filter on the value of the search string.
    The s => s.Title.Contains() code above is a Lambda Expression. Lambdas are
    used in method-based LINQ queries as arguments to standard query operator
    methods such as the Where method or Contains (used in the code above). LINQ
    queries are not executed when they are defined or when they are modified by
    calling a method such as Where, Contains or OrderBy. Rather, query execution
    is deferred. That means that the evaluation of an expression is delayed until
    its realized value is actually iterated over or the ToListAsync method is called.

    Note: The Contains method is run on the database, not in the c# code shown above.
    The case sensitivity on the query depends on the database and the collation. On SQL
    Server, Contains maps to SQL LIKE, which is case insensitive. In SQLite, with the
    default collation, it's case sensitive.

    If we change the signature of the Index method to have a parameter named id, the id
    parameter will match the option {id} placeholder for the default routes set in Startup.cs.
    That means that we can change the name of the parameter in the Index method
    and instead of having to type the name of the parameter in the URL segment, we can straight
    away write the string of the movie(s) we are looking for.

    However, we can't expect users to modify the URL every time they want to search for a movie.
    So now we'll add UI to help them filter movies.

    We add the UI by modifying the Index.cshtml file and add some <form> markup:
    \lstset{style=sharpc}
    \begin{lstlisting}
        <form asp-controller="Movies" asp-action="Index">
            <p>
                Title: <input type="text" name="SearchString">
                <input type="submit" value="Filter"/>
            </p>
        </form>
    \end{lstlisting}

    The HTML form tag uses the Form Tag Helper, so when we submit the form,
    the filter string is posted to the Index action of the controller.
    There is no [HttpPost] overload of the Index method as we might expect.
    We don't need it, because the method isn't changing the state of the app,
    just filtering data. We could add a HTTP Post overload of the Index method,
    but it wouldn't do much benefit in this case, especially if you would
    like to share the search results with other people.

    We can enforce the HTTP GET method, by adding the following parameter in
    the form Tag Helper: method="get".

    \chapter{Adding validation}
    In this section we will learn how to add validation to models, and ensure
    that the validation rules are enforced every time a user creates or edits
    data.

    \section{Keeping things DRY}
    One of the design tenets of MVC is DRY ("Don't repeat yourself"). ASP.NET
    MVC encourages developers to specify functionality or behaviour only once,
    and then have it reflected everywhere in an app. This reduces the amount
    of code we need to write and makes the written code less error prone,
    easier to test and easier to maintain.

    The validation support provided by MVC and EF Core Code First is a good
    example of the DRY principle in action. We can declaratively specify
    validation rules in one place (in the model class) and the rules are
    enforced everywhere in the app.

    \section{Adding validation rules to a model}
    DataAnnotations provides a built-in set of validation attributes that we
    apply declaratively to any class or property. (It also contains formatting
    like DataType that help with formatting and don'r provide any validation.)

    In the DataAnnotations, we have useful validation attributes such as Required,
    StringLength, RegularExpression and Range.

    The validation attributes specify behaviour that we want to enforce on the model
    properties they are applied to. The Required and MinimumLength attributes indicate
    that a property must have a value; but nothing prevents a user from entering white
    spaces to satisfy this validation. The RegularExpression attribute is used to limit
    what characters can be input. The Range attribute constrains a value within a
    specified range. The StringLength attribute lets us set the maximum length of a string
    property, and optionally its minimum length. Value types (such as decimal, int, float,
    DateTime) are inherently required and don't need the [Required] attribute.

    Having validation rules automatically enforced by ASP.NET helps make our app
    more robust. It also ensures that we can't forget to validate something and
    inadvertently let bad data into the database.

    \section{Validation Error UI in MVC}
    If we try to fill out a form with some invalid values, as soon as jQuery client side
    validation detects the error, it displays an error message.
    The form will automatically render an appropriate validation error message in each
    field containing an invalid value. The errors are enforced both client-side (using
    JavaScript and jQuery) and server-side(in case a user has JavaScript)
    disabled.

    A significant benefit is that we didn't need to change a single line of code in the
    rest of the code in order to enable this validation UI. The controller and views
    automatically pick up the validation rules that we specify by using validation
    attributes on the properties of the model.

    The form data is not sent to the server until there are no client side validation
    errors.

    \section{How validation works}
    We might wonder how the validation UI was generated without any updates to the code
    in the controller or views. In the POST controller methods, the ModelState.IsValid is
    called to check whether there are any validation errors, that is sent done by checking
    that the data corresponds to what is stated in the Model, bringing the responsibility
    of checking the data back to it, and not requiring any changes in the Controller code.

    In the view, when using the Input Tag Helpers we enforce validation with the
    asp-validation-for attribute. When rendering the page, the app fetches the validation
    information from the model, therefore making it unnecessary to change any code in the
    view.

    What's really nice about this approach is that neither the controller nor the view know
    anything about the actual validation rules being enforced or about the specific error
    messages. The validation rules and the error strings are specified only in the model.
    These same validation rules are automatically applied to any views that we might create
    that somehow edit the model.

    When we need to change validation logic, we can do so in exactly one place by adding
    validation attributes to the model. We won't have to worry about different parts of
    the application being inconsistent with how the rules are enforced - all validation logic
    will be defined in one place and used everywhere. This keeps the code very clean, and makes
    it easy to maintain and evolve. And it means that we'll be fully honoring the DRY principle.

    \section{Using DataType Attributes}
    The System.ComponentModel.DataAnnotations namespace provides formatting attributes in addition
    to the built-in set of validation attributes. The DataType attributes only provide hints for the
    view engine to format data (and supply attributes such as <a> for URLs and mailto for emails).
    We can use the RegularExpression attribute to validate the format of the data. The DataType
    attribute is used to specify a data type that is more specific than the database intrinsic
    type, they are not validation attributes. In this case we only want to keep track
    of the data, not the time. The DateType enumeration provides for many data types, such as Date,
    Time, PhoneNumber, Currency, EmailAddress and more. The DateType attribute can also enable the
    application to automatically provide type-specific features. For example, a mailto: link can be
    created for DataType.EmailAddress, and a date selector can be provided for DataType.Date in
    browsers that support HTML5. The DataType attributes emits HTML 5 data- attributes that HTML 5
    browsers can understand. The DataType attributes do not provide validation.

    DataType.Date does not specify the format of the data that is displayed. By default, the data
    field is displayed according to the default formats based on the server's CultureInfo.

    The DisplayFormat attribute is used to explicitly specify the date format.
    The ApplyFormatInEditMode setting specifies that the formatting should also be applied when the
    value is displayed in a text box for editing.
    We can use the DisplayFormat attribute by itself, but it's generally a good idea to use the
    DataType attribute. The DataType attribute conveys the semantics of the data as opposed
    to how to render it on a screen, and provides the following benefits that we don't get with
    DisplayFormat:
    \begin{itemize}
        \item The browser can enable HTML5 features (calendar control, the locale-appropriate
        currency symbol, email links, etc.)
        \item By default, the browser will render data using the correct format based on the
        locale
        \item The DataType attribute can enable MVC to choose the right field template to render
        the data (the DisplayFormat is used by itself uses the string template)
    \end{itemize}

    Note: jQuery validation does not work with the Range attribute and DateTime when used together.

    We will need to disable jQuery date validation to use the Range attribute with DateTime. It's generally
    not a good practice to compile hard dates in our models, so using the Range attribute and DateTime
    is discouraged.

    \chapter{Examining the Delete methods}
    There are two method related to deleting an entry from the database, the Delete and DeleteConfirmed
    methods that are scaffolded from the Models. Using again the movie database example, here is what they
    look like:

    \lstset{style=sharpc}
    \begin{lstlisting}
        //GET: Movies/Delete/5
        public async Task<IActionResult> Delete(int? id)
        {
            if(id == null)
            {
                return NotFound();
            }

            var movie = await _context.Movie
                .SingleOrDefaultAsync(m => m.ID == id);
            if(movie == null)
            {
                return NotFound();
            }

            return View(movie);
        }

        //POST: Movies/Delete/5
        [HttpPost, ActionName("Delete")]
        [ValidateAntiForgeryToken]
        public async Task<IActionResult> DeleteConfirmed(int id)
        {
            var movie = await _context.Movie
                SingleOrDefaultAsync(m => m.ID == id);
            _context.Movie.Remove(movie);
            await _context.SaveChangesAsync();
            return RedirectToAction("Index");
        }
    \end{lstlisting}

    Note that the HTTP GET Delete method doesn't delete the specified movie,
    it returns a view of the movie where we can submit (HttpPost) the deletion.
    Performing a delete operation in response to a GET request opens up a
    security hole.

    The [HttpPost] method that deletes the data is named DeleteConfirmed to give
    the HTTP POST method a unique signature or name. The CLR requires overload
    methods to have a unique parameter signature. However, here we have two Delete
    methods -- one for GET and one for POST -- that both have the same parameter
    signature.

    There are two approaches to this problem, one is to give the methods different
    names. That's what the scaffolding mechanism did in the preceding example.
    However, this introduces a small problem: ASP.NET maps segments of a URL to
    action methods by name, and if we rename a method, routing normally wouldn't
    be able to find that method. The solution is what we see in the example, which
    is to add the ActionName("Delete") attribute to the DeleteConfirmed method. That
    attribute performs mapping for the routing system so that a URL that includes /Delete/
    for a POST request will find the DeleteConfirmed method.

    Another common work around for methods that have identical names and signatures is
    to artificially change the signature of the POST method to include an extra (unused)
    parameter.
\end{document}